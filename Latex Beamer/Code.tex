%%%%%%%%%%%%%%%%%%%%%%%%%%%%%%%%%%%%%%%%%
% Beamer Presentation
% LaTeX Template
% Version 1.0 (10/11/12)
%
% This template has been downloaded from:
% http://www.LaTeXTemplates.com
%
% License:
% CC BY-NC-SA 3.0 (http://creativecommons.org/licenses/by-nc-sa/3.0/)
%
%%%%%%%%%%%%%%%%%%%%%%%%%%%%%%%%%%%%%%%%%

%----------------------------------------------------------------------------------------
%	PACKAGES AND THEMES
%----------------------------------------------------------------------------------------

\documentclass{beamer}

\mode<presentation> {

% The Beamer class comes with a number of default slide themes
% which change the colors and layouts of slides. Below this is a list
% of all the themes, uncomment each in turn to see what they look like.

%\usetheme{default}
%\usetheme{AnnArbor}
%\usetheme{Antibes}
%\usetheme{Bergen}
%\usetheme{Berkeley}
%\usetheme{Berlin}
%\usetheme{Boadilla}
%\usetheme{CambridgeUS}
%\usetheme{Copenhagen}
%\usetheme{Darmstadt}
%\usetheme{Dresden}
%\usetheme{Frankfurt}
%\usetheme{Goettingen}
%\usetheme{Hannover}
%\usetheme{Ilmenau}
%\usetheme{JuanLesPins}
%\usetheme{Luebeck}
\usetheme{Madrid}
%\usetheme{Malmoe}
%\usetheme{Marburg}
%\usetheme{Montpellier}
%\usetheme{PaloAlto}
%\usetheme{Pittsburgh}
%\usetheme{Rochester}
%\usetheme{Singapore}
%\usetheme{Szeged}
%\usetheme{Warsaw}

% As well as themes, the Beamer class has a number of color themes
% for any slide theme. Uncomment each of these in turn to see how it
% changes the colors of your current slide theme.

%\usecolortheme{albatross}
%\usecolortheme{beaver}
%\usecolortheme{beetle}
%\usecolortheme{crane}
%\usecolortheme{dolphin}
%\usecolortheme{dove}
%\usecolortheme{fly}
%\usecolortheme{lily}
%\usecolortheme{orchid}
%\usecolortheme{rose}
%\usecolortheme{seagull}
%\usecolortheme{seahorse}
%\usecolortheme{whale}
%\usecolortheme{wolverine}

%\setbeamertemplate{footline} % To remove the footer line in all slides uncomment this line
%\setbeamertemplate{footline}[page number] % To replace the footer line in all slides with a simple slide count uncomment this line

%\setbeamertemplate{navigation symbols}{} % To remove the navigation symbols from the bottom of all slides uncomment this line
}

\usepackage{graphicx} % Allows including images
\usepackage{booktabs} % Allows the use of \toprule, \midrule and \bottomrule in tables

%----------------------------------------------------------------------------------------
%	TITLE PAGE
%----------------------------------------------------------------------------------------

\title[Short title]{Matrix Project } % The short title appears at the bottom of every slide, the full title is only on the title page

\author{EE17BTECH11018 \& EE17BTECH11019} % Your name
\institute[IIT Hyderabad] % Your institution as it will appear on the bottom of every slide, may be shorthand to save space
{
IIT Hyderabad \\ % Your institution for the title page
\medskip
\textit{EE 1390-INTRO to AI and ML} 
}
\date{\today} % Date, can be changed to a custom date

\begin{document}

\begin{frame}
\titlepage % Print the title page as the first slide
\end{frame}




\begin{frame}
\frametitle{Geometry Question}
Let P be a point on $ y^2=8x $ which is at minimum distance from the centre C of the circle $x^2+y^2+12y=1$ .Find the equation of circle passing through C and has centre P  .\\~\\


\end{frame}

\begin{frame}
\frametitle{Figure}
\begin{figure}
\includegraphics[width=0.8\linewidth]{problemstatement[13479].png}
\end{figure}
\end{frame}

%------------------------------------------------

\begin{frame}
\frametitle{Matrix Equivalent of Question}
Circle equation is  $x^2+y^2+12y=1$ i.e A=1 ,B=0, C=1, D=0,E=12,F=-1
Comparing it with original conic equation $$(x^T)Vx+2(u^T)x+F=0$$
we get V=$ \left[ 
  \begin{array}{ c c }
     1& 0\\
     0 & 1
  \end{array} \right]$ and u=$ \left[
  \begin{array}{ c }
     0 \\
     6
  \end{array} \right]$
 .Therefore  the equation of circle is as follows:
\[
x^T \left[
  \begin{array}{ c c }
     1 & 0 \\
     0& 1
  \end{array} \right]
x+ \left[
  \begin{array}{ c c }
     0 & 12 \\
     
  \end{array} \right]
x=1
\]
Similarly parabola equation is 
\[ x^T \left[
  \begin{array}{ c c }
     0 & 0\\
     0 & 1
  \end{array} \right]
x+ \left[
  \begin{array}{ c c }
     -8&0 \\
    
  \end{array} \right]
\]
where x and V are Matrices .V is 2*2 and x is 2*1.

\end{frame}

%------------------------------------------------

\begin{frame}
\frametitle{Solution in MATRIX form}
\begin{block}{Idea}
We are given that circle's centre C is at a minimum distance from point P on the parabola i.e Circle is also at a minimum distance from the point on the parabola.
\newline
Therefore the normal drawn at the point P on the parabola should also be a normal to the circle (Since the tangents are parallel) i.e the normal at P passes through C .
\end{block}
We know tangent for the conic  $x^{T}Vx+2u^{T}x+F=0$ is $$(p^TV+u^T)x+p^Tu+F=0$$
Here p is $\left[
  \begin{array}{ c c }
     2t^2\\
     4t
  \end{array} \right],
V=\left[ 
  \begin{array}{ c c }
     0& 0\\
     0 & 1
  \end{array} \right],
  u= \left[ 
  \begin{array}{ c  }
     -4\\
     0
  \end{array} \right]$



\end{frame}


\begin{frame}
\frametitle{Solution in MATRIX form}
Therefore equation of tangent at P is 
\[
\left[ 
  \begin{array}{ c c }
     2t^{2}&4t
     
  \end{array} \right ].\left[ 
  \begin{array}{ c c }
     0 & 0\\
     0 & 1
  \end{array} \right]+\left[ 
  \begin{array}{ c c }
     -4&0
  \end{array} \right]))x+\left[ 
  \begin{array}{ c c }
     2t^2&4t
     
  \end{array} \right].\left[ 
  \begin{array}{ c }
     -4\\
     0 
  \end{array} \right]=0
  \]
 \\  Simplifying , we get
 \[
  \left[ 
  \begin{array}{ c c }
     -4 &4t
  \end{array} \right]x=8t^2
  \]
Given tangent 
\[
\left[ 
  \begin{array}{ c c }
     a& b
  \end{array} \right]x=c
  \]
  ,\\we can write normal equation $\left[ 
  \begin{array}{ c c }
     b & -a
  \end{array} \right]x=c^{\prime}$
  ,Therefore the equation of normal passing through p is as follows
  \begin{equation*}
  \left[
  \begin{array}{c c}
  t&1\\
  \end{array}
  \right]x=c^{\prime}
  \end{equation*}

\end{frame}

%---
%------------------------------------------------

\begin{frame}
\frametitle{Solution in MATRIX form}
Now substitute $x=c$ which is centre of circle  $c=\left[ 
  \begin{array}{ c  }
     0\\
    -6
  \end{array} \right]$
  \[
  \left[ 
  \begin{array}{ c c }
     t&1
  \end{array} \right]\cdot \left[ 
  \begin{array}{ c }
     0\\
    -6
  \end{array} \right]=c^{\prime}
  \text{ that is  }  c^{\prime}=-6.
  \]
  Therefore the equation of normal is
  \[\left[ 
  \begin{array}{ c c }
     t&1
  \end{array} \right]x=-6.
  \]
  Now substituting point P in the normal equation we get 
  \[
  \left[ 
  \begin{array}{ c c }
     t&1\\
  \end{array} \right]\cdot \left[ 
  \begin{array}{ c }
     2t^2\\
     4t
  \end{array} \right]=-6
  \]
  \text{ i.e } $2t^3+4t=-6$
Therefore $t=-1$ is the only real root for the above equation 
Therefore the point P is 
$$p=\left[ 
  \begin{array}{ c  }
     2\\
    -4
  \end{array} \right]
 $$

\end{frame}

%------------------------------------------------

\begin{frame}
\frametitle{Solution in MATRIX form}
Now taking P as centre and C as point on the circle ,we calculate radius of the required circle
\begin{gather*}
|p-c|^{2}=(2-0)^{2}+(-4+6)^{2} \\
r^2 = 2^{2} + 2^{2} = 8
\end{gather*}
Now, taking this radius and P as a centre, equation of the circle is,
\begin{gather*}
|x-p|^{2} = r^{2}\\
\left( x-p \right)\cdot (x-p)^{T} = r^{2} = 8\\
xx^{T} - xp^{T} - px^{T} + pp^{T} = 8
\end{gather*}
\end{frame}

\begin{frame}{Solution in MATRIX Form}
\begin{equation*}
\begin{aligned}
xx^{T} - x \left[\begin{array}{ c }
     2 -4\\
  \end{array} \right]
  -  \left[
  \begin{array}{ c }
     2\\
    -4
  \end{array} \right]x^{T} + 2^2 + (-4)^{2} = 8\\
\qquad \qquad xx^{T} - x \left[\begin{array}{ c }
     2 -4\\
  \end{array} \right]
  -  \left[
  \begin{array}{ c }
     2\\
    -4
  \end{array} \right]x^{T} + 12 = 0
  \end{aligned}
\end{equation*}
\centering
This is the equation of the circle required.
 
 
 
\end{frame}



\begin{frame}
\frametitle{Figure}
\begin{figure}
\includegraphics[width=0.8\linewidth]{Figure_1[13474].png}
\end{figure}
\end{frame}





\begin{frame}
\Huge{\centerline{The End}}
\end{frame}

%----------------------------------------------------------------------------------------

\end{document}
